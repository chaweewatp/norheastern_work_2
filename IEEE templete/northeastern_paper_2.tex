\documentclass[conference]{IEEEtran}
\IEEEoverridecommandlockouts
% The preceding line is only needed to identify funding in the first footnote. If that is unneeded, please comment it out.
\usepackage{cite}
\usepackage{amsmath,amssymb,amsfonts}
\usepackage{algorithmic}
\usepackage{graphicx}
\usepackage{textcomp}
\usepackage{xcolor}

\usepackage{multirow}
\usepackage{float}

\def\BibTeX{{\rm B\kern-.05em{\sc i\kern-.025em b}\kern-.08em
    T\kern-.1667em\lower.7ex\hbox{E}\kern-.125emX}}
\begin{document}

\title{Estimate Power generation of invisible solar site using State estimation\\
% {\footnotesize \textsuperscript{*}Note: Sub-titles are not captured in Xplore and
% should not be used}
\thanks{PEA, AIT, NEU}
}

\author{\IEEEauthorblockN{1\textsuperscript{st} Pornchai Chaweewat}
\IEEEauthorblockA{\textit{EECC} \\
\textit{AIT)}\\
Pathumthani, Thailand \\
chaweewat.p@gmail.com}
\and
\IEEEauthorblockN{2\textsuperscript{nd} Weerakorn Ongsakul}
\IEEEauthorblockA{\textit{EECC} \\
\textit{AIT)}\\
Pathumthani, Thailand \\
email address}
\and
\IEEEauthorblockN{3\textsuperscript{rd} Jai Govind Singh}
\IEEEauthorblockA{\textit{EECC} \\
\textit{AIT)}\\
Pathumthani, Thailand \\
email address}
\and
\IEEEauthorblockN{4\textsuperscript{th} Ali abur}
\IEEEauthorblockA{\textit{EEC} \\
\textit{NEU}\\
Boston, MA, USA \\
email address}
}

\maketitle

\begin{abstract}
The number of large scale and rooftop scale solar photovoltaic (PV) systems in electricity grid is growing at exponential rate.
Invisible solar PV refers to small-scale and rooftop solar sites in distribution system that are not mornitored by utilities.
These invisible solar PV sites cause challenging to utilies and system operators.
In this paper, a methodology is proposed to identify location and estimate installation capacity of solar PV sites.
With measurement voltage at any buses and power flow in any lines, state estimation (SE) algorithm could find power consumption in any buses.
The consumption are fed to find correlation with solar irradaiton data.
Thus, the correlative between these data could locate the site of invisible solar PV sites.
Futhermore, the change point method and proposed algorithm could estimate installation capacity of invisible solar PV sites.
The proposed methodology is test in distribution system i.e, four bus system and CIGRE Medium voltage distribution network with PV and Wind DER with randomly invsisble solar PV sites location.
The overall results of the study shows that the correlation between consumption and solar irradiation can perfectly identify invisible solar PV sites with $\text{F}_{1}$ score aboved 0.96 and Matthews correlation coefficient (MCC) score aboved 0.90.
The installation capacity estimation performe good with mean absolute error (MAPE) below 17\%.
\end{abstract}

\begin{IEEEkeywords}
  Solar photovoltaic, invisible solar PV site, Gaussian noise, state estimation, correlation, change point detection, Matthews correlation coefficient
\end{IEEEkeywords}


\section{Introduction}


Sixty years ago, the price of solar panels was astronomical.  The price was over 1,000 US dollars per watt in today's money with 1 percent efficiency~\cite{b18}.
As the cost continues to drop, solar panel systems are becoming much more viable for the average household.
Today's price is less than 2.8 US dollars per watt for solar panel system including solar module, inverter, wiring hardware, labor cost, and maintenance cost~\cite{b19}.
As result of exponentially acceralate increasing in number of globally solar PV integration, especailly in residential sector.


Many researchers have studied the impacts and risks of PV on distribution systems \cite{b20}-~\cite{b25}.
However, the detection and monitoring of residential PV systems has not been the focus of the studies and related research.

Recently, the highest penetration of solar PV, recognized a large number of unauthorized solar PV installation \cite{b26}.
Unauthorized solar PV installation creates safety risks and lack of visibility may result in incorrect planning, and operation, including over-voltage, back-feeding.
Futhermore, in worst case scenario, it may damange system equipment such as transformers, voltage regulators, as well as customer applicants \cite{b15}, \cite{b16}.
In previous studies, there are various reasons for unauthorized or incorrectly registered PV system:
a) owner decided not to apply for a permit to avoid fees \cite{b17},
b) regulations were required after the system was installed,
c) lack of awareness by the owner of diverse permitting rules,
d) difference rules depending on size and type of PV installation can make the owners believe they do not need a permit
e) change in property ownership including transfer,
f) multiple system installed or future iaddition at the same premise,
g) incorrect third party handling of the permit appication,
h) data entry and data maintenance errors.





The rest of the paper is structured as follows: Section II reviews related research on state estimation application for power system and previous detection and estimation techniques on invisible solar PV site.
Section III formulates the mathenatic model of measurement data as well as the structure of proposed method.
Section IV shows the results of proposed method on test cases.
Section V conclueds the paper and dicusses future research oppotunities on PV system detection and estimation.

\section{Literature reviews}

\subsection{Use of State estimation in power system}

\subparagraph{previous solar PV detection techniques}

\section{Problem Formulation}
  This section will presents the overall methodology of the study which will be devided into four parts.
  The conceptual of methodology is illustrated in Fig.~\ref{fig.method}.

  \begin{figure}[h!]
    \center
    \includegraphics[scale=0.5]{images/conceptual_methodology.png}
    \caption{conceptual methodology}
    \label{fig.method}
  \end{figure}

  \subsection{Measurement devices, measured data and accuracy}
    In this study, there are three type of measurements; bus, line, and transformers measurement.
    Voltage magnitude in per unit is measurement at bus. Active and reactive power flow, and magnitude curreng are measured at line and trasformers.
    Bus measurement has 1$\%$ error and line measurement has 3$\%$ error.

    To generate measurement data for testing prurposes, measurement error was added to tha actual measurements as shown in Equation~(\ref{eq.Z}).
    \begin{equation}
      Z=Z_{a} \pm +e_{z}
    \label{eq.Z}
    \end{equation}
    where $Z_{a}$ is actual data and $e_{z}$ is error added base on accuarcy of the measurement.

    These error are assumed to be modeled independent Gaussian random variable\cite{b2}.
    where where the error value is expected value from gaussian distribution.
    noise is guassian distribution,as shown in Equation~(\ref{eq.gaussian}).
    \begin{equation}
      g(x)=\frac{1}{\sigma \sqrt{2\pi}}e^{-\frac{1}{2}((x-\mu)/\sigma)^2}
    \label{eq.gaussian}
    \end{equation}

  \subsection{Load allocation based state estimation}

    SE is based on the weighted least square (WLS) approach\cite{b1}.
    The method solves the following WLS problem to obtain an estimate ofthe system operating point defined by the system state x:
    \begin{equation}
      \underset{x}{min} J(x)=\sum_{i=1}^{m}w_{i}(z_{i}-h_{i}(x))^{2}=\big[ z-h(x)\big]^{T}W\big[z-h(x)\big]
    \label{eq.jacobian}
    \end{equation}
    where $w_{i}$ and $h_{i}(x)$ represent the weight and the measurements function associated with measurement $z_{i}$, respectively.
    For the solution of this problem the conventional iterative method is adape by solving following normal equations at each iteration, to compute the update $x^{k+1}=x^{k}+\Delta x^{k}$.

    \begin{equation}
      \big[ G(x^{k}) \big] \Delta x^{k} = H^{T}(x^{k})W \big[ z-h(x^{k}) \big]
    \label{eq.delta_x}
    \end{equation}

    Where

    \begin{equation}
      G(x)=H^{T}(x)WH(x)
    \label{eq.gain}
    \end{equation}

    is the gain matrix and H is the jacobian of the measurement function $h(x)$.

  \subsection{Solar PV site's location identification}

    This part will descript location identification of invisible solar PV's site.
    The correlation-based feature selection (CFS) is key tool here.
    The basis of the CFS that was introduced in \cite{b27}.
    The CFS will be peformed as a statistical measure of relationship between solar irradiation and load allocation from previous part.
    The measure is best used in these two variables that demenstrate a linear relationship between each other.
    The correlation coefficient that indicates the straength of the relationship between these two variables can be found using following formula:

    \begin{equation}
      \text{r}_{IP} =\frac{\sum(I_{i}-\overline{I})(P_{i}-\overline{P})}{\sqrt{\sum(I_{i}-\overline{P})^{2}(P_{i}-\overline{P})^{2})}}
    \label{eq.corr}
    \end{equation}

    where $\text{r}_{IP}$ is the correlation coefficient of the linear relationship between the solar irradiation (I) and load consumption (P),
    $I_{i}$ is the values of the solar irradiation variable in a sample, $\overline{I}$ is the mean of $I$,
    $P_{i}$ is the values of the load consumption variable in a sample, $\overline{P}$ is the mean of $P$.

    The negaive correlation shows that the variables tend to move in opposity directions (i.e., when one variable increases, the other variable decreases).
    The probability of solar PV site located at bus $i$ is expressed in Equation~(\ref{eq.prob_i})
    \begin{equation}
      \text{prob}_{i}= =
      \begin{cases}
        0 & \text{if  $r_{i}>=0$} \\
        1 & \text{if  $r_{i}<0$}
      \end{cases}
    \label{eq.prob_i}
    \end{equation}

  \subsection{Estimate solar PV site's installation capacity}

  Once, we can identify location of invisible solar PV located. Then, we try to estimate size of invisble solar PV units.
  We deploy change-point detection of change-point analysis.
  The change-point detection algorithm is a powerfull tools used to detect abrupt changes in time series data.
  We add solar irradiation and load allocation where invisble solar PV located. Then, we can find chaning point of the dataset.

  \begin{equation}
    \Delta D_{i,t} = p_{e}\Delta I_{i,t}
  \end{equation}
  where $p_{e}$ is the size of the unauthorized PV system at bus $i$, and time $t$.

  Then, we remove noise and find mean, range or other to estimate solar capacity.

  \subsection{Evaluation}
  This section will descripe evaluation methods will being used in this paper.
  There are two part of evaluate the proposed invisible solar PV size.
  First, the identification of the invisible solar PV location is evaluated by confusion matrix, $text{F}_1$ score, and Matthews correlation coefficient (MCC).
  Second, the estimation of installation of invisible solar PV size is evaluated by mean absolute error (MAE), and mean absolute percentage errot (MAPE).

  The results of location of invisible solar PV detection will be represented in confusion matrix as exampled in Table~\ref{tab.confusion_matrix}

  \begin{table}[ht]
    \caption{Confusion matrix}

  \begin{tabular}{cc|c|c|}
  \cline{3-4}
                                                                                                        &                                                                           & \multicolumn{2}{c|}{True condition}                                                                                           \\ \cline{2-4}
  \multicolumn{1}{c|}{}                                                                                 & Total population                                                          & Condition positive                                            & Condition negative                                            \\ \hline
  \multicolumn{1}{|c|}{\multirow{2}{*}{\begin{tabular}[c]{@{}c@{}}Prediction\\ condition\end{tabular}}} & \begin{tabular}[c]{@{}c@{}}Prediction\\ condition\\ positive\end{tabular} & \begin{tabular}[c]{@{}c@{}}True positive\\ (TP)\end{tabular}  & \begin{tabular}[c]{@{}c@{}}False positive\\ (FP)\end{tabular} \\ \cline{2-4}
  \multicolumn{1}{|c|}{}                                                                                & \begin{tabular}[c]{@{}c@{}}Prediction\\ condition\\ negative\end{tabular} & \begin{tabular}[c]{@{}c@{}}False negative\\ (FN)\end{tabular} & \begin{tabular}[c]{@{}c@{}}True negative\\ (TN)\end{tabular}  \\ \hline
  \end{tabular}
  \label{tab.confusion_matrix}
  \end{table}
  where TP, TN, FP, and FN are defined as number of hit, correct rejection, false alarm and miss, respectively.

  \subsubsection{$\text{F}_{1}$ score}
  The $\text{F}_{1}$ score is a measure of a test's accuracy.
  The $\text{F}_{1}$ score is the harmonic average of the precision and recall, where an $\text{F}_{1}$ score reaches its best value at 1 (perfect precision) and worst at 0.
  The $\text{F}_{1}$ is formulated in Equation~(\ref{eq.F_1_score}).
  \begin{equation}
    \text{F}_{1}=\frac{2\times \text{TP}}{2\times \text{TP} + \text{FP} + \text{FN}}
  \label{eq.F_1_score}
  \end{equation}

  \subsubsection{Matthews correlation coefficient}
  The Matthews correlation coefficient (MCC) is First introduced by biochememist Brian W. Matthews in 1975 \cite{b28}.
  The MCC is essence an correlation coefficient between theobserved and predicted binary classification; its returns a value between -1 and +1.
  A coefficient of +1 represetnes a perfection prediction, 0 no better than random prediction and -1 indicates total disagreement between prediction and observation.
  An Equation~(\ref{eq.MCC}) shows the formular of MCC for 2 classes. The dervations are express in \cite{b29}.
  \begin{equation}
    \text{MCC}=\frac{\text{TP} \times \text{TN} - \text{FP} \times \text{FN}}{\sqrt{(\text{TP}+\text{FP})(\text{TP}+\text{FN})(\text{TN}+\text{FP})(\text{TN}+\text{FN})}}
  \label{eq.MCC}
  \end{equation}

  \subsubsection{Mean absolute error and Mean absolute percentage error}
  The MAE defines as the difference between the actual and estimated invisible solar PV capacity which is computed by Equation~(\ref{eq.MAE}).
  The MAPE expresses the scaled difference between the actual and estimatedd invisible solar PV capacity as a percentage of the actual invisible solar PV capacity.
  MAPE is scale independent and it can be used to compare estimation performance across different data sets. MAPE can be calculated by Equation~(\ref{eq.MAPE}).

  \begin{equation}
    \text{MAE} = \frac{1}{N}\sum_{i=1}^{N} \left| p_{a}-p_{e} \right|
  \label{eq.MAE}
  \end{equation}

  \begin{equation}
    \text{MAPE} = \frac{100}{N}\sum_{i=1}^{N} \left| \frac{p_{a}-p_{e}}{p_{a}} \right|
  \label{eq.MAPE}
  \end{equation}

  where $p_{a}$ and $p_{e}$ are the actual and estimated solar PV installation capacity. $N$ is the number of sample.

\section{Test cases and Results}

  Two power system are tested. There two test cases are based on four bus system and CIGRE Task Force C6.04.02 paper\cite{b3}.

  \subsection{four bus system}
    The one line diagram of the four bus system is shown in Fig.~\ref{fig.4_bus_system}.

    \begin{figure}[H]
      \center
      \includegraphics[scale=0.5]{images/four_bus_system.png}
      \caption{four bus system}
      \label{fig.4_bus_system}
    \end{figure}

    Table \ref{tab.Confusion_Matrix_4bus} shows the results. From the table, \ref{tab.Confusion_Matrix_4bus} it can be seen that the proposed method can predict perfectly location of invisible solar PV.
    \begin{table}[H]
      \centering
      \caption{Results of invisible solar PV identification on 4 bus system}
      \begin{tabular}{cc|c|c|}
        \cline{3-4}
                                                                                                                &     & \multicolumn{2}{l|}{Actual PV located} \\ \cline{3-4}
                                                                                                                &     & Yes                & No                \\ \hline
        \multicolumn{1}{|l|}{\multirow{2}{*}{\begin{tabular}[c]{@{}l@{}}Prediction \\ PV located\end{tabular}}} & Yes & 126                & 0                \\ \cline{2-4}
        \multicolumn{1}{|l|}{}                                                                                  & No  & 9                  & 60                \\ \hline
      \end{tabular}
      \label{tab.Confusion_Matrix_4bus}
    \end{table}
    The results is tested by $\text{F}_{1}$ score and MCC using Equation~\ref{eq.F_1_score}, \ref{eq.MCC}.
    The $\text{F}_{1}$ score is 0.965. The MCC is 0.901
    The false negative (FN) occurs when invisible solar PV's capacity is 1-2 kW where maximum demand is around 30 kW.

    \begin{table}[H]
      \centering
      \caption{Results of invisible solar PV estimation on 4 bus system}
      \begin{tabular}{ccc}
        \hline
        Size of invisible solar PV (kW) & MAE(kW) & MAPE (\%) \\
        \hline
        1-10                  & 0.77            & 13.9            \\
        11-20                 & 2.4             & 16.9            \\
        21-30                 & 3.75            & 14.2            \\
        31-40                 & 4.42            & 12              \\
        41-50                 & 3.85            & 8.4\\
        \hline
      \end{tabular}
      \label{tab.Error_4bus}
    \end{table}
    The overall MAPE is 13.587 \% and MAE is 2.919 kW.



  \subsection{CIGRE system}

    \begin{figure}[H]
      \center
      \includegraphics[scale=0.25]{images/CIGRE_network.png}
      \caption{CIGRE network}
      \label{fig.CIGRE_network}
    \end{figure}

    Table \ref{tab.Confusion_Matrix_CIGRE} shows the results. From the table, \ref{tab.Confusion_Matrix_4bus} it can be seen that the proposed method can predict perfectly location of invisible solar PV.

    \begin{table}[H]
      \centering
      \caption{Results of invisible solar PV identification on CIGREsystem}
      \begin{tabular}{cc|c|c|}
        \cline{3-4}
                                                                                                                &     & \multicolumn{2}{l|}{Actual PV located} \\ \cline{3-4}
                                                                                                                &     & Yes                & No                \\ \hline
        \multicolumn{1}{|l|}{\multirow{2}{*}{\begin{tabular}[c]{@{}l@{}}Prediction \\ PV located\end{tabular}}} & Yes & 154                & 8                 \\ \cline{2-4}
        \multicolumn{1}{|l|}{}                                                                                  & No  & 3                  & 159                \\ \hline
      \end{tabular}
      \label{tab.Confusion_Matrix_CIGRE}
    \end{table}

    The results is tested by $\text{F}_{1}$ score and MCC using Equation~\ref{eq.F_1_score}, \ref{eq.MCC}.
    The $\text{F}_{1}$ score is 0.965. The MCC is 0.9325.
    The false negative (FN) occurs when invisible solar PV's capacity is 10-20 kW.

    \begin{table}[H]
      \centering
      \caption{Results of invisible solar PV estimation on CIGRE system}
      \begin{tabular}{ccc}
        \hline
        Size of invisible solar PV (kW) & MAE (kW) & MAPE (\%) \\
        \hline

        1-100                  & 9.01            & 17.60            \\
        101-200                 & 24.3            & 15.97            \\
        201-300                & 50.00            & 19.85            \\
        301-400               & 52.55            & 13.98             \\
        401-500              & 54.92            & 12.28 \\
        \hline

      \end{tabular}
      \label{tab.Error_CIGRE}
    \end{table}

    The overall MAPE is 16.27 \% and MAE is 37.58 kW.

\section{Conclusion}
Here is Conclusion.
\section*{Acknowledgment}

\begin{thebibliography}{00}
  \bibitem{b18} https://www.theguardian.com/sustainable-business/2016/jan/31/solar-power-what-is-holding-back-growth-clean-energy
  \bibitem{b19} https://blog.pickmysolar.com/the-price-of-a-solar-panel-system-over-the-years
  \bibitem{b20} M. E. Baran, H. Hooshyar, Z. Shen, and A. Huang, “Accommodating high PV penetration on distribution feeders,” IEEE Trans. Smart Grid, vol. 3, no. 2, pp. 1039–1046, Jun. 2012.
  \bibitem{b21}  M. J. Reno, K. Coogan, R. J. Broderick, J. Seuss, and S. Grijalva, “Impact of PV variability and ramping events on distribution voltage regulation equipment,” in Proc. IEEE Photovolt. Spec. Conf., Tampa, FL, USA, 2014.
  \bibitem{b22} M. E. Baran, H. Hooshyar, Z. Shen, and A. Huang, “Accommodating high PV penetration on distribution feeders,” IEEE Trans. Smart Grid, vol. 3, no. 2, pp. 1039–1046, Jun. 2012.
  \bibitem{b23} P. Li, X. Yu, J. Zhang, and Z. Yin, “The H∞ control method of grid- tied photovoltaic generation,” IEEE Trans. Smart Grid, vol. 6, no. 4, pp. 1670–1677, Jul. 2015.
  \bibitem{b24}  A. Samadi, L. Söder, E. Shayesteh, and R. Eriksson, “Static equivalent of distribution grids with high penetration of PV systems,” IEEE Trans. Smart Grid, vol. 6, no. 4, pp. 1763–1774, Jul. 2015.
  \bibitem{b25}  H. Ravindra et al.,“Impact of PV on distribution protection system,” in Proc. North Amer. Power Symp. (NAPS), Champaign, IL, USA, Sep. 2012, pp. 1–6.
  \bibitem{b26} P. Fairley. (Jan. 2015). Hawaii’ s Solar Push Strains the Grid. [online]. Available: http://www.technologyreview.com/news/534266/ hawaiis-solar-push-strains-the-grid/
  \bibitem{b15} W. Staff. (Sep. 2014). Heco Customers Asked to Disconnect Unauthorized PV Systems. [Online]. Available: http://khon2.com/2014/ 09/05/heco-customers-asked-to-disconnect-unauthorized-pv-systems/
  \bibitem{b16} V. Schwarzer and R. Ghorbani, “Transient over-voltage mitiga- tion: Explanation and mitigation options for inverter-based dis- tributed generation projects,” Elect.Vehicle Transp. Center, Sch. Ocean Earth Sci. Technol., Univ. Hawai’ i Manoa, Honolulu, HI, US A , Tech. Rep. HNEI-02-15, Feb. 2014.
  \bibitem{b17} https://www.snopes.com/fact-check/is-it-illegal-florida-power-home-solar-storm/
  \bibitem{b1} Roytelman I. and S.M. Shahidehpour, ‘State estimation for electric power distribution systems in quasi real-time conditions’, IEEE Trans. Power Systems, 1993 winter meeting, paper no:090-1-PW RD.
  \bibitem{b2} A. Abur and A. G. Exposito, Power System State Estimation: Theory and Implementation. New York: Marcel Dekker, 2004.
  \bibitem{b3} CIGRE' Task Force C6.04.02: Developing benchmark models for integrating distributed energy resourcesS. Barsali, Kai Strunz, Zbigniew StyczynskiPublished 2005
  \bibitem{b4} STATE ESTIMATION FOR LOAD ALLOCATION IN DISTRIBUTION POWER SYSTEMS, R. Sharifian 1 E. Jafari 1 M. Rahimi 2 P. Ghaebi Panah 3
  \bibitem{b5} A Revised Branch Current-Based Distribution System State Estimation Algorithm and Meter Placement Impact, Haibin Wang, Student Member, IEEE, and Noel N. Schulz, Senior Member, IEEE
  \bibitem{b6} An Integrated Load AllocatiodState Estimation Approach for Distribution Networks, JorgePereira, J. T. Saraiva,Member, IEEE, V. Miranda, Member, IEEE
  \bibitem{b7} Distribution System State Estimation Using AMI Data, Mesut Baran, T. E. McDermott
  \bibitem{b8} Gu Chaojun, Student Member, IEEE, Panida Jirutitijaroen, Senior Member, IEEE, and Mehul Motani, Member, IEEE, Detecting False Data Injection Attacks in AC State Estimation
  \bibitem{b9} A Review on Distribution System State Estimation, Anggoro Primadianto and Chan-Nan Lu, Fellow, IEEE
  \bibitem{b10} Nathan Wallace, Stanislav Ponomarev, and Travis Atkison, Identification of Compromised Power System State Variables
  \bibitem{b11} X. Zhang and S. Grijalva, “A data-driven approach for detection and estimation of residential pv installations,” IEEE Transactions on Smart Grid, vol. 7, no. 5, pp. 2477–2485, Sept 2016.
  \bibitem{b12} H. Shaker, H. Zareipour, and D. Wood, “Estimating power generation of invisible solar sites using publicly available data,” IEEE Transactions on Smart Grid, vol. 7, no. 5, pp. 2456–2465, Sept 2016.
  \bibitem{b13} H. Jiang, X. Dai, D. W. Gao, J. J. Zhang, Y. Zhang, and E. Muljadi, “Spatial-temporal synchrophasor data characterization and analytics in smart grid fault detec- tion, identification, and impact causal analysis,” IEEE Transactions on Smart Grid, vol. 7, no. 5, pp. 2525– 2536, Sept 2016.
  \bibitem{b14} Detection and Estimation of the Invisible Units Using Utility Data Based on Random Matrix Theory, Xing He, Robert C. Qiu, Fellow, IEEE, Lei Chu, Qian Ai, Zenan Ling, Jian Zhang
  \bibitem{b27} M. A. Hall, “Correlation-based feature subset selection for machine learning,” Ph.D. dissertation, Dept. Comput. Sci., Univ. Waikato, Hamilton, New Zealand, 1999.

\end{thebibliography}






\end{document}
