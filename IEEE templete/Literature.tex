\section{Literature reviews}

\subsection{Applications of State estimation in power system}



State estimation on distribution power system is reviewed in \cite{b9}.

State estimation provides the optimum estimate of power system state based on received measurement units and the knowledge of network modeling. The measurement units may include as follows:
\begin{itemize}
  \item power injection (real/reactive),
  \item power flow (real/reactive),
  \item bus voltage magnitude,
  \item phase angle,
  \item line current magnitude,
  \item current injection magnitude.
\end{itemize}

State Estimation provides various benefits for power system analysis as follows:

\subsubsection{Real time mornitoring}

For power system in transmission and distribution levels~\cite{b5}. The Advance Meter infrastructure (AMI) data are fed into SE to real-time mornitor distribution power system in\cite{b7}. It improved the accuracy comparing to pseudo measurment based on historical load.

\subsubsection{Load allocation}
In distribution power system, SE assigns active and reactive power values to bus downstream\cite{b6}. Class curves and individual average demand of clients improved accuracy of load allocation\cite{b4}.  The data from AMR system, which are used as the pseudo-measurements, was used to estimate load on the distribution transformers.\cite{b5}.

\subsubsection{Bad data analysis}

In several real application, it happens that some flow measurement has no sign or the sign is wrong.
A re-weighting technique was used to reduced the weights association to these measurements\cite{b6}.
It also could helps to treat of power flow with erroneous sign.
Futhurmore, SE was implemented in cyber attack events~\cite{b10} on power system states. A detection false data injection method is proposed in \cite{b8}.

\subsubsection{Topology indentification}

the status of branches and switches are indentified in\cite{b6}.


\subsection{previous solar PV detection techniques}
The previous studies has been done on detection of location and estimation of power generation of invisible solar PV sites.
Reference \cite{b11} proposed a change-point detection algorithm for a time series for residential solar PV detection. These method required smart meter data and historical load profile.
The uncertainty of solar PV site and the data generated by a small set of selected representative sites were taken into account to estimates the power gneneration of known solar PV site in~\cite{b12}.  The new hybrid k-means and PCA techniques provided good accuracy of estimating invisible solar PV sites.
The big data characterization of smart grids and two-layer dynamuc optimal synchrophasor measurement devices selection algorithm for fault detection, identification, and causal impact analysis was proposed in~\cite{b13}.
The utility data based on random matrix therory (RMT) are used in detection and estimation of invisible solar PV in real case in China in\cite{b14}.

\subsection{Contributions}
Many researchers have studied the detection and estimation location of invisible solar PV installation~\cite{b11}-\cite{b14}.
These study depended on smart meter data, data set of solar PV generation, as well as historical demand of customer in each load.
This paper proposed method which identify location using state estimation, correlation measured and estimating the installation capacity using change point method of invisible solar PV site.
All solar PV's site is assumed to operate at unity power factor.
The solar PV is located randomly at any bus excepted slack bus.
The measurement units are located at optimum locations.
The measured data such as solar irradiation, power flow measurements in lines and voltage measurements at busses are generated with gaussian noise.
The proposed method consists of four steps:
\begin{itemize}
  \item collect solar irradiation, measurement data
  \item allocate power demend in each buses, nodes.
  \item indentify invisible solar PV site
  \item estimate solar PV site's installation capacity
\end{itemize}
