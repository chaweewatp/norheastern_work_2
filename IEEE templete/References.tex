\begin{thebibliography}{00}
  \bibitem{b18} https://www.theguardian.com/sustainable-business/2016/jan/31/solar-power-what-is-holding-back-growth-clean-energy
  \bibitem{b19} https://blog.pickmysolar.com/the-price-of-a-solar-panel-system-over-the-years
  \bibitem{b20} M. E. Baran, H. Hooshyar, Z. Shen, and A. Huang, “Accommodating high PV penetration on distribution feeders,” IEEE Trans. Smart Grid, vol. 3, no. 2, pp. 1039–1046, Jun. 2012.
  \bibitem{b21}  M. J. Reno, K. Coogan, R. J. Broderick, J. Seuss, and S. Grijalva, “Impact of PV variability and ramping events on distribution voltage regulation equipment,” in Proc. IEEE Photovolt. Spec. Conf., Tampa, FL, USA, 2014.
  \bibitem{b22} M. E. Baran, H. Hooshyar, Z. Shen, and A. Huang, “Accommodating high PV penetration on distribution feeders,” IEEE Trans. Smart Grid, vol. 3, no. 2, pp. 1039–1046, Jun. 2012.
  \bibitem{b23} P. Li, X. Yu, J. Zhang, and Z. Yin, “The H∞ control method of grid- tied photovoltaic generation,” IEEE Trans. Smart Grid, vol. 6, no. 4, pp. 1670–1677, Jul. 2015.
  \bibitem{b24}  A. Samadi, L. Söder, E. Shayesteh, and R. Eriksson, “Static equivalent of distribution grids with high penetration of PV systems,” IEEE Trans. Smart Grid, vol. 6, no. 4, pp. 1763–1774, Jul. 2015.
  \bibitem{b25}  H. Ravindra et al.,“Impact of PV on distribution protection system,” in Proc. North Amer. Power Symp. (NAPS), Champaign, IL, USA, Sep. 2012, pp. 1–6.
  \bibitem{b26} P. Fairley. (Jan. 2015). Hawaii’ s Solar Push Strains the Grid. [online]. Available: http://www.technologyreview.com/news/534266/ hawaiis-solar-push-strains-the-grid/
  \bibitem{b15} W. Staff. (Sep. 2014). Heco Customers Asked to Disconnect Unauthorized PV Systems. [Online]. Available: http://khon2.com/2014/ 09/05/heco-customers-asked-to-disconnect-unauthorized-pv-systems/
  \bibitem{b16} V. Schwarzer and R. Ghorbani, “Transient over-voltage mitiga- tion: Explanation and mitigation options for inverter-based dis- tributed generation projects,” Elect.Vehicle Transp. Center, Sch. Ocean Earth Sci. Technol., Univ. Hawai’ i Manoa, Honolulu, HI, US A , Tech. Rep. HNEI-02-15, Feb. 2014.
  \bibitem{b17} https://www.snopes.com/fact-check/is-it-illegal-florida-power-home-solar-storm/
  \bibitem{b1} Roytelman I. and S.M. Shahidehpour, ‘State estimation for electric power distribution systems in quasi real-time conditions’, IEEE Trans. Power Systems, 1993 winter meeting, paper no:090-1-PW RD.
  \bibitem{b2} A. Abur and A. G. Exposito, Power System State Estimation: Theory and Implementation. New York: Marcel Dekker, 2004.
  \bibitem{b3} CIGRE' Task Force C6.04.02: Developing benchmark models for integrating distributed energy resourcesS. Barsali, Kai Strunz, Zbigniew StyczynskiPublished 2005
  \bibitem{b4} STATE ESTIMATION FOR LOAD ALLOCATION IN DISTRIBUTION POWER SYSTEMS, R. Sharifian 1 E. Jafari 1 M. Rahimi 2 P. Ghaebi Panah 3
  \bibitem{b5} A Revised Branch Current-Based Distribution System State Estimation Algorithm and Meter Placement Impact, Haibin Wang, Student Member, IEEE, and Noel N. Schulz, Senior Member, IEEE
  \bibitem{b6} An Integrated Load AllocatiodState Estimation Approach for Distribution Networks, JorgePereira, J. T. Saraiva,Member, IEEE, V. Miranda, Member, IEEE
  \bibitem{b7} Distribution System State Estimation Using AMI Data, Mesut Baran, T. E. McDermott
  \bibitem{b8} Gu Chaojun, Student Member, IEEE, Panida Jirutitijaroen, Senior Member, IEEE, and Mehul Motani, Member, IEEE, Detecting False Data Injection Attacks in AC State Estimation
  \bibitem{b9} A Review on Distribution System State Estimation, Anggoro Primadianto and Chan-Nan Lu, Fellow, IEEE
  \bibitem{b10} Nathan Wallace, Stanislav Ponomarev, and Travis Atkison, Identification of Compromised Power System State Variables
  \bibitem{b11} X. Zhang and S. Grijalva, “A data-driven approach for detection and estimation of residential pv installations,” IEEE Transactions on Smart Grid, vol. 7, no. 5, pp. 2477–2485, Sept 2016.
  \bibitem{b12} H. Shaker, H. Zareipour, and D. Wood, “Estimating power generation of invisible solar sites using publicly available data,” IEEE Transactions on Smart Grid, vol. 7, no. 5, pp. 2456–2465, Sept 2016.
  \bibitem{b13} H. Jiang, X. Dai, D. W. Gao, J. J. Zhang, Y. Zhang, and E. Muljadi, “Spatial-temporal synchrophasor data characterization and analytics in smart grid fault detec- tion, identification, and impact causal analysis,” IEEE Transactions on Smart Grid, vol. 7, no. 5, pp. 2525– 2536, Sept 2016.
  \bibitem{b14} Detection and Estimation of the Invisible Units Using Utility Data Based on Random Matrix Theory, Xing He, Robert C. Qiu, Fellow, IEEE, Lei Chu, Qian Ai, Zenan Ling, Jian Zhang
  \bibitem{b27} M. A. Hall, “Correlation-based feature subset selection for machine learning,” Ph.D. dissertation, Dept. Comput. Sci., Univ. Waikato, Hamilton, New Zealand, 1999.

\end{thebibliography}
