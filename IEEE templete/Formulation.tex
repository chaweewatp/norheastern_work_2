\section{Problem Formulation}
This section will presents the overall methodology of the study which will be devided into four parts.
The conceptual of methodology is illustrated in Fig.~\ref{fig.method}.

\begin{figure}[h!]
  \center
  \includegraphics[scale=0.5]{images/conceptual_methodology.png}
  \caption{conceptual methodology}
  \label{fig.method}
\end{figure}

\subsection{Measurement devices, measured data and accuracy}
In this study, there are three type of measurements; bus, line, and transformers measurement.
Voltage magnitude in per unit is measurement at bus. Active and reactive power flow, and magnitude curreng are measured at line and trasformers.
Bus measurement has 1$\%$ error and line measurement has 3$\%$ error.

To generate measurement data for testing prurposes, measurement error was added to tha actual measurements as shown in Equation~\ref{eq.Z}.
\begin{equation}
  Z=Z_{a} \pm +e_{z}
\label{eq.Z}
\end{equation}
where $Z_{a}$ is actual data and $e_{z}$ is error added base on accuarcy of the measurement.

These error are assumed to be modeled independent Gaussian random variable\cite{b2}.
where where the error value is expected value from gaussian distribution.
noise is guassian distribution,as shown in Equation~\ref{eq.gaussian}
\begin{equation}
  g(x)=\frac{1}{\sigma \sqrt{2\pi}}e^{-\frac{1}{2}((x-\mu)/\sigma)^2}
  \label{eq.gaussian}
\end{equation}

\subsection{Load allocation based state estimation}

SE is based on the weighted least square (WLS) approach\cite{b1}.
The method solves the following WLS problem to obtain an estimate ofthe system operating point defined by the system state x:
\begin{equation}
  \underset{x}{min} J(x)=\sum_{i=1}^{m}w_{i}(z_{i}-h_{i}(x))^{2}=\big[ z-h(x)\big]^{T}W\big[z-h(x)\big]
\label{eq.jacobian}
\end{equation}
where $w_{i}$ and $h_{i}(x)$ represent the weight and the measurements function associated with measurement $z_{i}$, respectively.
For the solution of this problem the conventional iterative method is adape by solving following normal equations at each iteration, to compute the update $x^{k+1}=x^{k}+\Delta x^{k}$.

\begin{equation}
  \big[ G(x^{k}) \big] \Delta x^{k} = H^{T}(x^{k})W \big[ z-h(x^{k}) \big]
\label{eq.delta_x}
\end{equation}

Where

\begin{equation}
  G(x)=H^{T}(x)WH(x)
\label{eq.gain}
\end{equation}

is the gain matrix and H is the jacobian of the measurement function $h(x)$.

\subsection{Solar PV site's location identification}




This part will descript location identification of invisible solar PV's site.
The correlation-based feature selection (CFS) is key tool here.
The basis of the CFS that was introduced in \cite{b27}.
The CFS will be peformed as a statistical measure of relationship between solar irradiation and load allocation from previous part.
The measure is best used in these two variables that demenstrate a linear relationship between each other.
The correlation coefficient that indicates the straength of the relationship between these two variables can be found using following formula:

\begin{equation}
  \text{r}_{IP} =\frac{\sum(I_{i}-\overline{I})(P_{i}-\overline{P})}{\sqrt{\sum(I_{i}-\overline{P})^{2}(P_{i}-\overline{P})^{2})}}
\label{eq.corr}
\end{equation}

where $\text{r}_{IP}$ is the correlation coefficient of the linear relationship between the solar irradiation (I) and load consumption (P),
$I_{i}$ is the values of the solar irradiation variable in a sample, $\overline{I}$ is the mean of $I$,
$P_{i}$ is the values of the load consumption variable in a sample, $\overline{P}$ is the mean of $P$.

The negaive correlation shows that the variables tend to move in opposity directions (i.e., when one variable increases, the other variable decreases).
The probability of solar PV site located at bus $i$ is expressed in Equation~\ref{eq.prob_i}
\begin{equation}
  \text{prob}_{i}= =
  \begin{cases}
    0 & \text{if  $r_{i}>=0$} \\
    1 & \text{if  $r_{i}<0$}
  \end{cases}
\label{eq.prob_i}
\end{equation}

\subsection{Estimate solar PV site's installation capacity}

Once, we can identify location of solar PV located. Then, we try to estimate size of invisble solar PV units.
We deploy change-point detection of change-point analysis.
The change-point detection algorithm is a powerfull tools used to detect abrupt changes in time series data.
We add solar irradiation and load allocation where solar PV located. Then, we can find chaning point of the dataset.


\begin{equation}
  \Delta D_{i,t} = p\Delta I_{i,t}
\end{equation}
where $p$ is the size of the unauthorized PV system at bus $i$, and time $t$.

Then, we remove noise and find mean, range or other to estimate solar capacity.
