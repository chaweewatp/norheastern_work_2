\section{Introduction}


Sixty years ago, the price of solar panels was astronomical.  The price was over 1,000 US dollars per watt in today's money with 1 percent efficiency~\cite{b18}.
As the cost continues to drop, solar panel systems are becoming much more viable for the average household.
Today's price is less than 2.8 US dollars per watt for solar panel system including solar module, inverter, wiring hardware, labor cost, and maintenance cost~\cite{b19}.
As result of exponentially acceralate increasing in number of globally solar PV integration, especailly in residential sector.

Recently, the highest penetration of solar PV, recognized a large number of unauthorized solar PV installation \cite{b26}.
Unauthorized solar PV installation creates safety risks and lack of visibility may result in incorrect planning, and operation, including over-voltage, back-feeding.
Futhermore, in worst case scenario, it may damange system equipment such as transformers, voltage regulators, as well as customer applicants \cite{b15}, \cite{b16}.
In previous studies, there are various reasons for unauthorized or incorrectly registered PV system
For example, owner decided not to apply for a permit to avoid fees \cite{b17}.
The difference rules depending on size and type of PV installation can make the owners believe they do not need a permit \cite{b30}.

Futhermore, many researchers have studied the impacts and risks of PV on distribution systems \cite{b20}-\cite{b25}.
These impacts can cause physical and financial damage to utility and solar PV owners, especially invisble solar PV which utility could not mornitor real-time.
However, the detection and estimation of residential PV systems has not been the focus of the studies and related research.
This issue is key motivation to this study.



The rest of the paper is structured as follows: Section II reviews related research on state estimation application for power system and previous detection and estimation techniques on invisible solar PV site.
Section III formulates the mathenatic model of measurement data as well as the structure of proposed methods as well as evaluation processes.
Section IV shows the results of proposed method on test cases.
Section V conclueds the paper and dicusses future research oppotunities on invisible PV system detection and estimation.
